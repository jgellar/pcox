\documentclass[12pt]{article}
%\documentclass[landscape, 10pt]{article}
\usepackage{geometry}        % See geometry.pdf to learn the layout options. There are lots.
\geometry{letterpaper}          % ... or a4paper or a5paper or ... 
%\geometry{landscape}        % Activate for for rotated page geometry
%\usepackage[parfill]{parskip}  % Activate to begin paragraphs with an empty line rather than an indent
\usepackage{graphicx}
\usepackage{amssymb}
\usepackage{epstopdf}
\usepackage{mathtools}
\usepackage{enumerate}
\usepackage{array}
\usepackage{enumitem}
\usepackage{bigstrut}
\usepackage{fullpage}
\usepackage{multirow}
\usepackage{authblk}
\usepackage[table]{xcolor}
\usepackage{arydshln}
\usepackage{setspace}
\usepackage[font={footnotesize}]{caption}
\usepackage[hang,flushmargin]{footmisc}
\DeclareGraphicsRule{.tif}{png}{.png}{`convert #1 `dirname #1`/`basename #1 .tif`.png}
%\setlength{\bibsep}{0pt plus 0.3ex}
\usepackage{natbib}
\bibpunct{(}{)}{;}{a}{,}{,}



\newcommand{\bphi}{\mbox{\boldmath $\phi$}}
\newcommand{\bpsi}{\mbox{\boldmath $\psi$}}
\newcommand{\bgamma}{\mbox{\boldmath $\gamma$}}
\newcommand{\btheta}{\mbox{\boldmath $\theta$}}
\newcommand{\btau}{\mbox{\boldmath $\tau$}}
\newcommand{\bc}{\mbox{\boldmath $c$}}
\newcommand{\bJ}{\mbox{\boldmath $J$}}
\newcommand{\bb}{\mbox{\boldmath $b$}}
\newcommand{\bg}{\mbox{\boldmath $g$}}
\newcommand{\bW}{\mbox{\boldmath $W$}}
\newcommand{\bZ}{\mbox{\boldmath $Z$}}
\newcommand{\bD}{\mbox{\boldmath $D$}}
\newcommand{\bU}{\mbox{\boldmath $U$}}
\newcommand{\bH}{\mbox{\boldmath $H$}}
\newcommand{\bV}{\mbox{\boldmath $V$}}
\newcommand{\be}{\begin{enumerate}}
\newcommand{\ee}{\end{enumerate}}
\newcommand{\bi}{\begin{itemize}}
\newcommand{\ei}{\end{itemize}}
\newcommand{\beq}{\begin{equation}}
\newcommand{\eeq}{\end{equation}}
\newcommand{\bea}{\begin{eqnarray}}
\newcommand{\eea}{\end{eqnarray}}
\newcommand{\beqa}{\begin{eqnarray*}}
\newcommand{\eeqa}{\end{eqnarray*}}
\newcommand{\bdm}{\begin{displaymath}}
\newcommand{\edm}{\end{displaymath}}
\newcommand{\nn}{\nonumber}


\linespread{1.6}
\renewcommand{\familydefault}{cmss}


\begin{document}

\title{\texttt{pcox} Package}
\author{Jonathan Gellar}
\date{}
\maketitle

\section{Introduction}

\texttt{pcox} is an R program to fit penalized Cox models with smooth effects of covariates, using a penalized
spline basis. These effects can either be smooth over the domain of the covariate, smooth over the time domain,
or both. Both scalar and functional predictors are supported. Functional predictors can either be
measured along their own domain, or along the same time domain as the hazard function. We refer to the former
type of predictor as a static functional covariate, and the latter as a time-varying covariate.

%
%Additionally, the functional predictors
%can either be measured along their own domain, or along the same time domain as the hazard function. We call
%the former a non-concurrent or baseline functional predictor, and the latter a concurrent functional predictor. Another
%way to view a concurrent functional predictor is as a densely-measured time-varying covariate. 

These definitions result in the following general model:
\begin{equation} \label{eqn:model}
	\log \lambda_i(t) = \log \lambda_0(t) + f(x_i, t) + g(X_i, t) + h(Z_i, t)
\end{equation}
In this notation $\lambda_i(t)$ is the hazard function for subject $i$ and $\lambda_0(t)$ is the baseline hazard function;
$x_i$, $X_i$, and $Z_i$ are scalar, static functional, and time-varying functional covariates, respectively; and
$f$, $g$, and $h$ are unknown functions we want to estimate. As proposed above this model is very general, but in practice we
will likely want to parameterize the way in which each covariate affects the hazard function. Table
\ref{tab:terms} summarizes the different parameterizations for $f$, $g$, and $h$, and the corresponding
syntax for the \texttt{pcox} formula. These terms will be discussed further in Section \ref{sec:covariate}.

%\begin{singlespace}
\begin{table}[t]
\centering
\caption{Summary of the types of terms allowed in an \texttt{pcox} formula.\label{tab:terms}}
\begin{tabular}{llll}
\bf Covariate Effect & \bf Time Effect & \bf Term Formula & \bf Implementation in \texttt{pcox}  \\[1ex]
\hline \\[-1.5ex]
\multicolumn{4}{l}{\bf Scalar Covariate Terms:} \\
Linear	& Static	& $\beta x$	& \texttt{x} \\
Linear	& Varying	& $\beta(t) x$	& \texttt{tv(x)} \\
Nonlinear	& Static	& $\beta(x)$	& \texttt{s(x)} \\
Nonlinear	& Varying	& $\beta(x,t)$	& \texttt{tv(s(x))} \\[2ex]
\multicolumn{4}{l}{\bf Time-Static Functional Covariate Terms:} \\
Linear	& Static	& $\int_{\mathcal{S}} \beta(s)X_i(s)\,ds$		& \texttt{lf(X)}, \texttt{lf.vd(X)} \\
Linear	& Varying	& $\int_{\mathcal{S}} \beta(s,t)X_i(s)\,ds$	& \texttt{tv(lf(X))}, \texttt{tv(lf.vd(X))} \\
Nonlinear	& Static	& $\int_{\mathcal{S}} \beta[s,X_i(s)]\,ds$		& \texttt{af(X)}\\
Nonlinear	& Varying	& $\int_{\mathcal{S}} \beta[s,t,X_i(s)]\, ds$	& \texttt{tv(af(X))} \\[2ex]
\multicolumn{4}{l}{\bf Time-Varying Functional Covariates as Concurrent Terms:} \\
Linear	& Static	& $\beta Z_i(t)$		& \texttt{Z} \\
Linear	& Varying	& $\beta(t) Z_i(t)$	& \texttt{tv(Z)} \\
Nonlinear	& Static	& $\beta[Z_i(t)]$		& \texttt{s(Z)} \\
Nonlinear	& Varying	& $\beta[Z_i(t), t]$		& \texttt{tv(s(Z))} \\[2ex]
\multicolumn{4}{l}{\bf Time-Varying Functional Covariates as Historical Terms:} \\
Linear	& Static	& $\int_{\delta(t)}^t \beta(s)Z_i(s)\,ds$	& \texttt{hlf(Z)} \\
Linear	& Varying	& $\int_{\delta(t)}^t \beta(s,t)Z_i(s)\,ds$	& \texttt{tv(hlf(Z))} \\
Nonlinear	& Static	& $\int_{\delta(t)}^t \beta[s, Z_i(s)]\,ds$	& \texttt{haf(Z)} \\
Nonlinear	& Varying	& $\int_{\delta(t)}^t \beta[s, t, Z_i(s)]\,ds$	& \texttt{tv(haf(Z))} \\[2ex]
\multicolumn{4}{l}{\bf Random Effects:} \\
\end{tabular}
\end{table}

The basic mechanism for modeling the nonparametric effects is to apply a penalized spline
basis to the functions $f$, $g$, and $h$ in \eqref{eqn:model}. The model is then estimated by
maximizing the penalized partial likelihood (PPL), 

\beq
	\ell^{(p)}_{\rho}(\btheta) = \sum_{i:\delta_i=1}\left\{ \eta_i(\btheta,t) -
		\log\left(\sum_{j: Y_j\geq Y_i}e^{\eta_j\left(\btheta,t\right)}\right)\right\} - \rho P(\btheta)
\eeq
where $\btheta$ are the model parameters (e.g., spline coefficients), $\eta_i(\btheta,t) = f(x_i,t) + g(X_i,t) + h(Z_i,t)$
is the linear predictor for subject $i$ at time $t$, $\rho$ is a smoothing parameter (possibly a vector),
and $P(\btheta)$ is an appropriate penalty on the parameters. For a given $\rho$, the PPL is maximized
by Newton-Raphson. The smoothing parameter(s) $\rho$ may be optimized by a number of different criteria: 
the cross-validated likelihood (CVL), a likelihood-based information criterion (AIC, AIC$_c$, or EPIC), maximum
likelihood (ML), or restricted maximum likelihood (REML). We intend to make each of these optimization criteria
available in \texttt{pcox}, except for the CVL, which is quite computationally intensive.

\texttt{pcox} is essentially a wrapper for four other \texttt{R} packages: \texttt{survival} and \texttt{coxme} for
fitting penalized Cox models, and \texttt{mgcv} and \texttt{refund} for processing the covariates. Section
\ref{sec:fitting} describes how the models are fit, and Section \ref{sec:covariate} describes how the
covariates are processed.

%\beqa
%	L^{(p)}(\btheta) &=& \prod_{i:\delta_i=1}\left\{ \frac{e^{\log \lambda_0(t_i) + \eta_i(\btheta,t)}}{\sum_{j: Y_j\geq Y_i}e^{\log h_0(t_i) + \eta_j(\btheta,t)}}\right\}
%		= \prod_{i:\delta_i=1} \left\{ \frac{e^{\eta_i(\btheta,t)}}{\sum_{j: Y_j\geq Y_i}e^{\eta_j(\btheta,t)}}\right\} \\
%\eeqa
%where $\btheta$ are the parameters (e.g., spline coefficients), and $\eta_i(\btheta,t) = $

\section{Model Fitting} \label{sec:fitting}

\subsection{Basic structure}

We take advantage of existing implementations of penalized Cox models in \texttt{R}. Two functions that fit these
models are the \texttt{coxph()} function from the \texttt{survival} package, and the \texttt{coxme()} function from the package
of the same name. The former optimizes the smoothing parameter using a user-defined control function, whereas
the latter is optimized specifically through ML or REML. \texttt{pcox} is essentially a wrapper for these two functions.
If the user chooses to optimize the model via ML or REML, \texttt{coxme()} is called, otherwise \texttt{coxph()} is
called.

\subsection{Specifying penalized terms in a \texttt{coxph} formula}

Instructions for defining penalized terms in a \texttt{coxph} formula are found in \citet{Grambsch1998}. The idea
is to create a \texttt{coxph.penalty} object, which is the model matrix corresponding to the term, with a number
of attributes defining the penalization. These attributes include a function to compute the first and second derivatives
of the penalty, the control function for the ``outer" loop to optimize the smoothing parameter, and parameters for these
functions. I have created the function \texttt{acTerm()} to convert an \texttt{mgcv} smooth term (created by calling
\texttt{smoothCon()} into a \texttt{coxph.penalty} object.

\subsection{Specifying penalized terms in a \texttt{coxme} formula}

\texttt{coxme} was designed to fit Gaussian frailty models, where the user enters the distribution of the random effects
by supplying a variance matrix. This variance corresponds to the inverse of the penalty matrix from a penalized
Cox model. Unfortunately, many of the common penalty matrices that we use are non-invertible. I am currently working
with the developer of \texttt{coxme}, Terry Therneau, to allow \texttt{coxme()} to accept the penalty matrix instead
of the variance matrix.


\section{Covariate Processing} \label{sec:covariate}

Penalized spline bases are implemented for generalized additive models very flexibly by the \texttt{mgcv} package; we
take advantage of this flexibility by allowing for \texttt{mgcv}-style model terms. This allows the choice of basis,
number of knots, and form of the penalty to be selected by the user.

Terms involving functional predictors can
be specified using certain terms from the \texttt{refund} package, which creates the appropriate \texttt{mgcv}-style
regression term.

Time varying covariates
can either be included as concurrent terms or historical terms. Concurrent terms allow only the value of the covariate
at time $t$ to impact the hazard at time $t$; this is the traditional way of handling time-varying covariates in a Cox
model. Historical terms allow the hazard at time $t$ to depend on the entire history of the covariate up to and including
time $t$.

\subsection{Time-varying effects}
Time varying effects can be specified as a \texttt{tv()} term.

\section{Examples}


\begin{singlespace}
\bibliographystyle{Chicago}
\scriptsize
\bibliography{/Users/jonathangellar/Documents/Bibtex/FunctionalRegression}
\end{singlespace}



\end{document} 